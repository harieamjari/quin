\documentclass{article}
\usepackage[T1]{fontenc}
%\usepackage[utf8]{inputenc}
\usepackage{ltablex}
\usepackage{pdflscape}
\usepackage{array}
\usepackage{algpseudocode}
\usepackage{hyperref}
\usepackage{enumitem}
\usepackage{amsmath}
\usepackage{amssymb}
\usepackage{ltablex}
\usepackage{fancyvrb}
\usepackage{alphabeta}

\usepackage[greek, english]{babel}
\usepackage[table]{xcolor} 
\makeatletter
\renewcommand\verbatim@font{\normalfont\fontencoding{T1}\ttfamily}
\makeatother
%\newunicodechar{Λ}{$\Lambda$}
%\newunicodechar{Ψ}{$\Psi$}
%\newunicodechar{Ξ}{$\Xi$}
%\newunicodechar{Φ}{$\Phi$}
%\newunicodechar{Δ}{$\Delta$}
%\newunicodechar{Σ}{\textgreek{Σ}}
%\newunicodechar{Γ}{\ensuremath{\Gamma}}
%\newunicodechar{Π}{\ensuremath{\Pi}}


%\newunicodechar{Λ}{\textgreek{Λ}}
%\newunicodechar{Ψ}{$\Psi$}
%\newunicodechar{Ξ}{$\Xi$}
%\newunicodechar{Φ}{$\Phi$}
%\newunicodechar{Δ}{$\Delta$}
%\newunicodechar{Σ}{\textgreek{Σ}}
%\newunicodechar{Γ}{\ensuremath{\Gamma}}
%\newunicodechar{Π}{\ensuremath{\Pi}}

\title{A quinary cipher for SMS encryption}
\author{\href{https://github.com/harieamjari}{Al-buharie}}


\begin{document}
\maketitle

\section*{Preface}

The previous year, I watched Veritasium's video about vulnerabilities
in our phone system where they briefly performed a demonstration of
an   SS7 attack and redirected an SMS message away from the intended
recipient. What suprise me the most was the lack of encryption and
they were able to read the message in its plaintext form as is. Several
months from that, and almost completely forgetting about it, I worked
on a (draft) research about remote access via SMS where I needed some kind
cryptographic functionality to address this insecure communication
channel. The result of that was a quinary cipher. Recognizing this can
be used for other purposes,
I decided to write a separate article about it.

I place a big disclaimer first, the ciphers I described within this
article are insecure and are not meant for public use. These are for
demonstrative purposes only and more of a "hack" than a real solution
to address SMS security for the public, but hey, this fit perfectly on
\textit{Hacker} News.

I've done great simplification and elided some of the mathematical
concepts to keep it simple and short, so if you want to dig further,
you can look up abstract algebra and finite fields where much of this
article spent explaining.

\medskip

With that said, it's time to transcend beyond binary.

\newpage

\begin{Verbatim}[fontsize=\small]
223 433 224 410 043 131 402 420 142 102 121 214 013 132 413  WÜpǧ)6ÅG3¤kPBT
413 104 210 211 030 020 444 032 243 414 404 044 422 002 443  Teì åñöAamhx¡2c
432 121 314 341 134 214 343 243 403 040 132 322 413 444 441  E¤l0tkbaOΛB?Tö1
320 334 412 024 120 340 032 404 203 202 442 023 233 222 343  üv;nØΨAhM4JUÖ>b
423 344 334 022 443 241 302 030 104 024 431 120 344 011 431  Yäv<c/5åen,Øäß,
324 401 214 042 201 243 233 333 423 300 322 403 021 334 042  qækFàaÖÑY¥?O#vF
141 134 233 134 244 412 024 043 113 131 340 322 331 444 202  .tÖtz;n§Q)Ψ?+ö4
123 302 030 043 443 234 403 114 132 114 442 114 212 400 223  V5å§cuOjBjJj9èW
403 141 141 324 320 030 244 101 142 410 131 134 442 434 224  O..qüåzΞGÇ)tJwp
213 242 333 042 420 134 421 330 201 343 134 322 402 310 023  RHÑFÅt'Φàbt?6òU
412 130 121 124 223 220 241 423 204 333 000 300 211 420 204  ;Δ¤oWø/YfÑ@¥ Åf
214 141 231 022 410 403 421 012 412 030 421 244 034 214 121  k.*<ÇO'7;å'zsk¤
303 031 404 242 203 414 440 231 104 101 421 410 440 304 311  N(hHMmΣ*eΞ'ÇΣg!
312 420 421 014 232 210 042 103 023 313 021 432 031 413 330  :Å'iCìFLUS#E(TΦ
031 242 042 101 423 221 004 210 302 331 302 143 203 141 333  (HFΞY%dì5+5¿M.Ñ
444 430 142 030 123 034 404 211 214 432 322 132 341 003 330  öΓGåVsh kE?B0KΦ
044 443 014 203 012 303 440 421 121 443 232 213 234 100 333  xciM7NΣ'¤cCRu£Ñ
440 200 244 240 332 442 143 332 341 343 010 144 344 104 230  Σ$zΠDJ¿D0béyäe_
044 020 024 424 033 244 214 333 234 134 301 014 434 410 423  xñnrZzkÑutÆiwÇY
440 213 440 212 434 031 220 223 002 112 122 024 414 130 244  ΣRΣ9w(øW28=nmΔz
212 404 013 431 231 034 024 442 442 313 333 411 404 030 120  9hP,*snJJSÑ"håØ
021 240 401 343 123 103 402 444 244 231 333 341 140 403 012  #ΠæbVL6öz*Ñ0ΩO7
111 313 102 314 331 103 041 404 240 224 441 411 143 301 104  ÉS3l+L-hΠp1"¿Æe
234 014 034 141 421 241 201 010 011 124 010 300 122 402 211  uis.'/àéßoé¥=6
300 023 234 004 104 043 223 231 414 010 330 321 004 123 120  ¥Uude§W*méΦ&dVØ
030 002 323 400 410 404 322 323 141 401 033 032 100 412 312  å2XèÇh?X.æZA£;:
112 244 344 020 010 333 133 442 130 441 411 430 244 302 034  8zäñéÑÄJΔ1"Γz5s
234 102 103 312 044 114 210 214 300 141 212 302 032 002 304  u3L:xjìk¥.95A2g
241 343 434 241 242 042 323 441 022 111 242 444 211 322 331  /bw/HFX1<ÉHö ?+
301 201 124 014 131 123 330 424 010 114 003 030 413 010 132  Æàoi)VΦréjKåTéB
144 314 224 131 432 400 034 202 222 312 213 430 020 112 311  ylp)Eès4>:RΓñ8!
341 003 203 003 100 114 101 000 130 134 100 134 044 444 130  0KMK£jΞ@Δt£txöΔ
034 100 320 300 310 011 410 100 013 013 410 013 404 444 413  s£ü¥òßÇ£PPÇPhöT
444 221 420 300 012 213 440 300 230 000 033 424 311 133 143  ö%Å¥7RΣ¥_@Zr!Ä¿
430 233 030 300 032 433 143 320 202 204 240 210 402 302 011  ΓÖå¥AÜ¿ü4fΠì65ß
320 140 134 120 010 112 412 412 433 123 303 312 032 113 100  üΩtØé8;;ÜVN:AQ£
032 014 013 412 001 011 241 241 243 312 330 331 203 211 310  AiP;Θß//a:Φ+M ò
003 201 401 341 200 101 124 124 124 331 233 033 120 321 131  Kàæ0$Ξooo+ÖZØ&)
332 442 421 224 132 434 210 042 320 331 002 144 304 410 012  DJ'pBwìFü+2ygÇ7
142 433 304 302 432 031 122 214 003 334 413 341 414 202 304  GÜg5E(=kKvT0m4g
232 121 004 340 231 334 014 141 042 430 012 043 104 042 331  C¤dΨ*vi.FΓ7§eF+
300 334 431 024 030 002 004 044 012 141 430 040 302 332 132  ¥v,nå2dx7.ΓΛ5DB
144 222 000 332 422 343 401 114 222 010 401 131 014 044 011  y>@D¡bæj>éæ)ixß
341 044 031 123 204 103 443 241 330 104 424 404 143 332 300  0x(VfLc/Φerh¿D¥
034 104 403 112 320 410 344 324 133 010 442 440 414 333 230  seO8üÇäqÄéJΣmÑ_
003 410 440 311 232 041 034 432 413 301 044 244 041 433 323  KÇΣ!C-sETÆxz-ÜX
441 102 432 301 104 411 402 233 420 034 341 342 420 332 034  13EÆe"6ÖÅs0IÅDs
212 043 412 140 103 022 042 143 434 100 000 343 200 100 304  9§;ΩL<F¿w£@b$£g
244 132 010 124 003 433 101 134 430 012 121 243 333 132 131  zBéoKÜΞtΓ7¤aÑB)
301 030 032 102 412 212 413 243 301 404 141 410 320 241 112  ÆåA3;9TaÆh.Çü/8
\end{Verbatim}

\newpage


\section{Introduction}

When writing a SMS text message, we usually use a limited
set of prededined characters. These set of allowable characters in
a standard SMS text message are defined in 3GPP TS
23.038. By default, if you use the GSM 7-bit default character
set, in theory, you can fit about 140 bytes or 160 characters in a single
SMS message but you can't actually make use of everything  because some
are control characters and are not meant to be displayed.

\medskip

\begin{tabular}{|c|c|c|c|c|c|c|c|c|}
\hline
              & \verb+0x00+ & \verb+0x10+ & \verb+0x20+ & \verb+0x30+ & \verb+0x40+ & \verb+0x50+ & \verb+0x60+ & \verb+0x70+ \\
\hline
  \verb+0x00+ & @ & $\Delta$ & \verb+SP+ & 0 & ¡ & P & ¿ & p  \\
\hline
  \verb+0x01+ & £ & \_ & ! & 1 & A & Q & a & q  \\
\hline
  \verb+0x02+ & \$ & $\Phi$ & " & 2 & B & R & b & r  \\
\hline
  \verb+0x03+ & ¥ & $\Gamma$ & \# & 3 & C & S & c & s  \\
\hline
  \verb+0x04+ & è & $\Lambda$ & ¤ & 4 & D & T & d & t  \\
\hline
  \verb+0x05+ & é & $\Omega$ & \% & 5 & E & U & e & u  \\
\hline
  \verb+0x06+ & ù & $\Pi$ & \& & 6 & F & V & f & v  \\
\hline
  \verb+0x07+ & ì & $\Psi$ & \textquotesingle & 7 & G & W & g & w  \\
\hline
  \verb+0x08+ & ò & $\Sigma$ & ( & 8 & H & X & h & x  \\
\hline
  \verb+0x09+ & Ç & $\Theta$ & ) & 9 & I & Y & i & y  \\
\hline
  \verb+0x0A+ & \verb+LF+ & $\Xi$ & * & : & J & Z & j & z  \\
\hline
  \verb+0x0B+ & Ø & \verb+ESC+ & + & ; & K & Ä & k & ä  \\
\hline
  \verb+0x0C+ & ø & Æ & , & < & L & Ö & l & ö  \\
\hline
  \verb+0x0D+ & \verb+CR+ & æ & - & = & M & Ñ & m & ñ  \\
\hline
  \verb+0x0E+ & Å & ß & . & > & N & Ü & n & ü  \\
\hline
  \verb+0x0F+ & å & É & / & ? & O & § & o & à \\
\hline
\end{tabular}%

\medskip

\begin{itemize}
  \item[] \verb+SP+ represents the space character.
  \item[] \verb+CR+ represents the carriage return character.
  \item[] \verb+LF+ represents the line feed character.
  \item[] \verb+ESC+ represents the escape character\footnote{An escape extension to other tables. See 3GPP TS 23.038.}

%\footnote{An escape extension to other tables, either the national language single shift table (if specified in the TP-UDH), or the GSM 7 bit default alphabet extension table. See 3GPP TS 23.038..}
\end{itemize}

See, SMS has a feature called, 
"single shift mechanism" which allows you to incorporate additional
characters (from the GSM 7-bit default alphabet extension table in our case) using the format
\verb+ESC<CHARACTER>+.
Unfortunately, if the receiving entity does not understand this escape mechanism,
the standard requires it be displayed as a space character and since the default extension table is not
fully populated and not all character has a representation when escaped,
we will be bound to run on these problems of spaces on our message. Not only that, it would be illegal to place \verb+ESC+ at the end of an SMS message without a following character.

We can however limit ourselves to a specific 125 characters from the default character set by excluding \verb+ESC+, \verb+CR+ and \verb+LF+;\footnote{We can choose other combination of \texttt{ESC}, \texttt{CR}, \texttt{SP} and \texttt{LF} as long as both the handset and network makes no modification to
these characters.}
it just so happens that 125 is a prime power of 5, perfect for creating a finite field. I will
shortly discuss this later.


\section{Basic One Time Pad (OTP)}

First, I wish to establish that "quin" (from the word "quinary") is a unit
of data that can represent either 0, 1, 2, 3, or 4; the same idea as "bit" which
can represent 0 or 1.

In binary, we usually perform a OTP using an XOR operation which happens to be equivalent
to modulus 2 arithmetic.

\medskip

\begin{table}[!hp]
\centering
\begin{tabular}{c|cc}
  $+$ & 0 & 1 \\
  \hline
  0 & 0 & 1 \\
  1 & 1 & 0 \\
\end{tabular}
  \caption{Addition table mod 2}
\end{table}

\medskip

This is basically a commutative group. Notice in the bottom-right quadrant, each elements occurs
only once for each rows and columns specially the identity element 0. This
allows us to be able to take an inverse and decrypt whatever ciphertext we
have. 1 is inverse of 1 and 0 is the inverse of 0. If we add 1 (plaintext)
and 1 (key), we get 0 (ciphertext). To decrypt it, we add the inverse
of our key which is 1 and we're back again to our plaintext 1.

%\clearpage

For our quinary OTP, we will be taking the same idea but modulus 5 instead.

\begin{table}[!hp]
\centering
  \begin{tabular}{c|ccccc}
  $+$ & 0 & 1 & 2 & 3 & 4\\
  \hline
  0 & \cellcolor{yellow!50}0 & 1 & 2 & 3 & 4 \\
  1 & 1 & 2 & 3 & 4 & \cellcolor{yellow!50}0\\
  2 & 2 & 3 & 4 & \cellcolor{yellow!50}0 & 1 \\
  3 & 3 & 4 & \cellcolor{yellow!50}0 & 1 & 2 \\
  4 & 4 & \cellcolor{yellow!50}0 & 1 & 2 & 3 \\
\end{tabular}
  \caption{Addition table mod 5}
\end{table}


This is also a commutative group where elements occur once
for each rows and
columns. I've colored the additive identity with yellow to
make it more visible.

If we add  4 and 2, we get 1.


\[ 4 + 2 = 1 \pmod 5 \]

If we add to both sides the inverse of 2 which is 3, we're back to
4.
\[ 4 = 4\pmod 5 \]

I hate to dig in further with the formal mathematics but you have
to remember that, kind of like numbers, you can add and subtract using these 5 elements
to end up with also one of those 5 element. And if you use 3 quins, you'd be able to represent
the 125 characters where the operation, \( + \), is performed component wise.

\[ (a_0,a_1,a_2) + (b_0,b_1,b_2) = (a_0+b_0,a_1+b_1,a_2+b_2)\]

Thus, $012+341 = 303$.

\subsection{Exercise}

Here is a OTP encrypted ciphertext. As an excercise, try decrypting it.\footnote{HINT: First convert both the Ciphertext and Key to it's quinary representation according to the table in the appendix. Then compute
the inverse of the key and add it to the ciphertext. Once you've
converted the result back to it's character representation, you shall have
your decrypted message}

\medskip

Ciphertext: \textquotesingle§EÆ\_SñBùpä

Key: uöVFÉ35P/Ç$\Psi$


\section{Finite fields}

In mathematics, you're probably used to dealing with inifinitely number of
numbers. The set which contain these infinitely many numbers is the
set $\mathbb{R}$ and this includes 0, 1, 1.2, -45.34, 82, 2839, 0.5, $\pi$, $e$
and much more, but dealing with mathematics doesn't necessarily mean
we are limited to these numbers to do computation and
arithmetic with. In fact you can do arithmetic with a finite set of numbers (or elements) and all rules of basic algebra will also apply. These rules are called
the field axioms:

\textbf{Field definition:} For a set \(F\) with two binary operations, \(+\) and \(\cdot\), to be a field, the following axioms must be satisfied for all elements \(a,b,c\in F\).

\begin{itemize}
    \item \textbf{Closure:} $a+b \in F$ and $a\cdot b \in F$.
    \item \textbf{Associativity:} $a+(b+c) = (a+b)+c$ and $a\cdot (b\cdot c) = (a\cdot b)\cdot c$.
    \item \textbf{Commutativity:} $a+b = b+a$ and $a\cdot b = b\cdot a$.
    \item \textbf{Distributivity:} $a\cdot (b +c) = a\cdot b + a\cdot c$.
    \item \textbf{Identity:} $\exists 0 \in F$ such that $a+0=a$ and $\exists 1 \in F$ such that $a\cdot 1=a$.
    \item \textbf{Inverse:} $\exists(-a),\in F$ such that $a+(-a) = 0$ and $\exists a^{-1}\in F, a \neq 0$ such that $a\cdot a^{-1} = 1$.
\end{itemize}

For example, the integers modulo 5 ($\mathbb{F}_5 = \{ 0, 1, 2, 3, 4 \}$)
is a field under addition and multiplication. 5 being a prime number
guarantees that multuplication of two non zero numbers is also not
zero. 

\begin{table}[!hp]
\centering
  \begin{tabular}{c|ccccc}
  $\cdot$ & 0 & 1 & 2 & 3 & 4\\
  \hline
  0 & 0 & 0 & 0 & 0 & 0 \\
  1 & 0 & \cellcolor{yellow!50}1 & 2 & 3 & 4\\
  2 & 0 & 2 & 4 & \cellcolor{yellow!50}1 & 3 \\
  3 & 0 & 3 & \cellcolor{yellow!50}1 & 4 & 2 \\
  4 & 0 & 4 & 3 & 2 & \cellcolor{yellow!50}1 \\
\end{tabular}
  \caption{Multiplication table mod 5}
\end{table}

But this would not be enough to represent all of the 125
characters. We have to use three quins instead.  There's a weird way of
mathematicians representing this and that they use polynomials. For our case, we'll
be using a 2nd degree polymomial with coefficients
in $F_5$:

\[ a_0 + a_1 x + a_2 x^2, a_n \in F_5 \]

To add two polynomials, group like-terms and add the coefficients mod 5.

\[ (a_0 + a_1 x + a_2 x^2) + (b_0 + b_1 x + b_2 x^2) = (a_0+b_0) + (a_1 + b_1)x + (a_2 + b_2)x^2 \]

And we can multiply two polynomials the way we already know:

\begin{align*}
  & (a_0 + a_1 x + a_2 x^2) (b_0 + b_1 x + b_2 x^2) \\
=& a_0(b_0 + b_1 x + b_2 x^2) + \\
  &  a_1 x(b_0 + b_1 x + b_2 x^2) + \\
   & a_2 x^2(b_0 + b_1 x + b_2 x^2) \\
=& a_0 b_0 + a_0 b_1 x + a_0 b_2 x^2 + \\
  & a_1 b_0 x + a_1 b_1 x^2 + a_1 b_2 x^3 + \\
  & a_2 b_0 x^2 + a_2 b_1 x^3 + a_2 b_2 x^4 \\
=& a_0 b_0 + \\ 
  & (a_0 b_1 + a_1 b_0) x + \\
  & (a_0 b_2 + a_1 b_1 + a_2 b_0)x^2 + \\
  & (a_1 b_2 + a_2 b_1) x^3 + \\
  & a_2 b_2 x^4
\end{align*}

Yikes, but this doesn't satisfy our properties of \textbf{Closure}. The
problem was that the multiplication of two 2 degrees polynomials resulted
into a 4th degree polynomial instead of another 2 degree polynomial. To
solve this we'll have to reduce it by dividing it with a 3rd degree
polynomial and taking the remainder as the result. But not just any
polynomial, a monic irreducible polynomial; if you choose the wrong one,
you won't have a system that satisfies the field axioms and all things
get awry.

Previously, we choose a prime number, 5, which can't have factors, as
the modulus to create a field with 5 elements. The reason stays the same
for choosing an irreducible polynomial (which also can't have factors)
for reducing our 4th degree polynomial.

It might be easy to show what what is reducible instead. For example, $x^2
+ 1$ is reducible in the set of polynomials with coefficient in $F_5$
(aka $F_5[x]$) because it can be factored to $(x+2)(x+3)$.  $x^3 +
2x^2 + 2x + 4$ is reducible in $F_5[x]$ because it can be factored
to $(x^2+2)(x+2)$. There are many ways of testing a polynomial's
irreducibility. One small quick test to check if it has linear factors is
by plugging in 1, 2, 3 or 4. If it has roots, then it has linear factors
and thus reducible. I'll spare the list of tests used in testing a
polynomial's irreducibility because it's not the scope of this article.

%but for our purpose, I'll use $ 3 + 4x^2 + x^3 $ as an example of an irreducible polynomial.

One big question you may have now is how do we even compute the multiplicative inverse given a polynomial
$p$? The answer is the Euclid's Division Lemma:

\[ pq + r = d \]

where $p$ is the divisor, $q$ is the quotient, $r$ is the remainder and $d$ is the dividend. If we
substitute $d$ for our modulus irreducible polynomial, then $d$ becomes 0 since $d \mod d = 0$. Rearranging
to find, $1/p$:

\begin{align*}
  pq +r &= 0 \\
  r &= -pq \\
  1 &= -pq/r \\
  1/p &= -q/r \\
  1/p &= q/(-r)
\end{align*}

This is a recursive function. To compute $1/p$, we compute the remainder $r$,
of $d/p$ by long division, and as we go further down the recursive call, $r$ will
get progressively become a constant polynomial (because $\textrm{deg}(r) < \textrm{deg}(p))$. If $r$ is non-zero, we have a solution and return the inverse according to the table in Table 3, if $r$ is zero, we don't have a solution (but that never
happens with an irreducible polynomial because it has no factors).

\begin{algorithmic}[1]
  \State $I \gets (0, 1, 3, 2, 4)$
  \Comment{Pre-initialized multiplicative inverses}
  \Function{MultiplicativeInverse}{p}
  \If{$p.\Call{Degree}{} == 0$}
    \State \Return $I_{p_0}$ 
    \Comment{Returns the multiplicative inverse of the 0th term in $p$}
  \EndIf
  \State $r \gets d \% p$
  \State $q \gets d / p$
  \State \Return $q\cdot \Call{MultiplicativeInverse}{-r}$
  \EndFunction
\end{algorithmic}

\section{Linear feedback shift register as a PRNG}

Consider the element 4. If you multiply 4 by itself mod 5, you get
1. If you multiply 1 by 4, you get 4 again. So by multiplying 4
again and again by itself, you can reach 2 elements, $\{1,4\}$. How about
2?

\begin{align*}
  2^2 &= 4 \mod 5\\
  2^3 &= 3 \mod 5 \\
  2^4 &= 1 \mod 5 \\
  2^5 &= 2 \mod 5 \\
  2^6 &= 4 \mod 5 && \textrm{We're back again to the beginning}
\end{align*}

So the element 2 generated 4 elements, \{1, 2, 3, 4\}. How about
3?

\begin{align*}
  3^2 &= 4 \mod 5 \\
  3^3 &= 2 \mod 5 \\
  3^4 &= 1 \mod 5 \\
  3^5 &= 3 \mod 5 \\
  3^6 &= 4 \mod 5 && \textrm{We're back again to the beginning}
\end{align*}

It also generated all elements except 0. If such element can generate
all elements of our multiplicative group, $\{1, 2, 3, 4\}$, then we call that element a primitive element.
Now, does such concept exists
in our polynomial multiplication? where you can multiply a polynomial
by itself again and again to reach all elements except 0? Yes there is.
In fact, we can multiply the polynomial $x$ by itself again and again
to reach all non zero elements by choosing a primitive polynomial as the
modulus in our field. Here we'll keep things simple with 25 elements then represent them as degree
1 polynonials $q_1 + q_2x, q_n \in \mathrm{F}_5$. Let's start multiplying: 

\begin{align*}
  x &= 0 + 1x && \textrm{$x$ is just $x$} \\
  x^2 &= 0 + 0x + x^2 && \textrm{Oops, our polynomial is now a 2nd degree.}
\end{align*}

Now we run into a bit of problem, we have to reduce it with a primitive polynomial
to make sure we satisfy our properties of Closure. I've choosen, $3 + 3x + x^2$ as an example of a primitive polynomial and if we take the remainder
of $x^2/(3 + 3x + x^2)$ we get, $2 + 2x$.  That means, $x^2 = 2 + 2x$.  Notice that if you rearrange the equation so that the right side becomes 0, you get, $3 + 3x + x^2 = 0$. Continuing
for $x^3$ and $x^4$:

\begin{align*}
  x^3 &= 2x + 2x^2 = 2x + 2(2 +2x) \rightarrow 4 + 1x \\
  x^4 &= 4x + 1x^2 = 4x + 1(2 +2x) \rightarrow 2 + 1x
\end{align*}

And
if we keep going, multiplying with $x$ and substituting the right side's $x^2$ with $2+2x$:

\begin{align*}
  x^{5} = 2 + 4x , x^{6} = 3 + 0x ,x^{7} = 0 + 3x ,x^{8} = 1 + 1x , \\
  x^{9} = 2 + 3x , x^{10} = 1 + 3x ,x^{11} = 1 + 2x ,x^{12} = 4 + 0x ,\\
  x^{13} = 0 + 4x, x^{14} = 3 + 3x ,x^{15} = 1 + 4x ,x^{16} = 3 + 4x ,\\
  x^{17} = 3 + 1x , x^{18} = 2 + 0x ,x^{19} = 0 + 2x ,x^{20} = 4 + 4x ,\\
  x^{21} = 3 + 2x , x^{22} = 4 + 2x ,x^{23} = 4 + 3x ,x^{24} = 1 + 0x ,\\
  x^{25} = 0 + 1x
\end{align*}

we can reach all possible permutation of $q_0 + q_1 x, \in F_5$ (except 0 that is). If we
used 126 quins, we'll be able to create a pseudorandom number generator
that has a tremendously long period. This PRNG is very insecure but it
suffices as a very basic toy stream cipher.

%\begin{align*}
%x ^{126}+ x ^{125}+ x ^{124}+ 3 x ^{123}+ 3 x ^{122}+ 4 x ^{121}+ 3 x ^{120}+ \\
%4 x ^{119}+ 2 x ^{118}+ 2 x ^{117}+ 4 x ^{115}+ 2 x ^{114}+ 2 x ^{113}+ 4 x ^{112}+ \\
%  3 x ^{110}+ 4 x ^{109}+ 2 x ^{108}+ x ^{107}+ 4 x ^{106}+ 3 x ^{105}+ x ^{104}+ \\
%  2 x ^{103}+ 2 x ^{102}+ x ^{101}+ 2 x ^{100}+ 3 x ^{99}+ 4 x ^{98}+ 3 x ^{97}+ \\
%  3 x ^{96}+ x ^{95}+ 3 x ^{94}+ 4 x ^{92}+ 3 x ^{91}+ 2 x ^{90}+ x ^{89}+ 2 x ^{88}+ \\
%  x ^{87}+ x ^{85}+ 4 x ^{82}+ 2 x ^{81}+ 3 x ^{80}+ 4 x ^{77}+ 4 x ^{74}+ 2 x ^{73}+\\
%  4 x ^{72}+ 3 x ^{71}+ 4 x ^{69}+ 2 x ^{67}+ x ^{65}+ 2 x ^{64}+ 3 x ^{63}+ \\
%  3 x ^{62}+ x ^{61}+ 4 x ^{60}+ 3 x ^{59}+ x ^{58}+ x ^{57}+ x ^{56}+ 2 x ^{55}+\\
%  2 x ^{54}+ 2 x ^{52}+ x ^{51}+ 2 x ^{50}+ 4 x ^{49}+ 3 x ^{47}+ 3 x ^{46}+ 2 x ^{42}+\\
%  x ^{40}+ 3 x ^{39}+ 4 x ^{38}+ x ^{37}+ 3 x ^{34}+ 3 x ^{33}+ 3 x ^{32}+ 2 x ^{31}+\\
%  3 x ^{29}+ 2 x ^{28}+ x ^{27}+ 2 x ^{26}+ 4 x ^{25}+ 3 x ^{24}+ 3 x ^{23}+ 3 x ^{22}+\\
%  3 x ^{21}+ x ^{20}+ x ^{19}+ 4 x ^{18}+ 3 x ^{17}+ 3 x ^{16}+ 2 x ^{15}+ 3 x ^{14}+\\
%  3 x ^{13}+ 2 x ^{12}+ 2 x ^{11}+ x ^{10}+ 2 x ^{7}+ x ^{6}+ 2 x ^{5}+ x ^{4}+ x ^{3}+\\
%  x ^{2}+ 3 x+ 3
%\end{align*}

%as the primitive polynomial.

\section{A toy quinary SPN cipher (TQC-135)}

So far, our addition based cipher is very primitive compared to the modern
ciphers we used today\footnote{I say modern, but AES which we use today
was create more than a decade ago, so might a bit misleading there.}.
More importantly it lacks Shannon's property of Diffusion where the 
statistical properties of the plaintext is spread over the ciphertext
(so a single quin change in the plaintext would change all the quins in
the ciphertext).

To address this, we'll be creating a very basic abstracted Substitution Permutation Network
based cipher. On SPNs, there are round functions, $R(m)$, which on its
own does not provide much good security for encrypting messages, but when
composed several N-times, $R(R(R(R(R(m)))))$, and introducing a RoundKey\footnote{RoundKeys
are generated from a Key Scheduler which takes in a Master Key and produces various keys
from it.}
for each every round, provides good confusion and diffusion properties. Although
what we're designing was an inspiration was from AES, much of the design
was scrapped for implementation convenience.

Now we have to set conditions for choosing the size of
our block cipher. Here are the list I've prepared:

\begin{enumerate}

\item The block cipher size must be a multiple of 3 and 5 so whole characters
can be encrypted without truncation and all 5 elements can appear to reduce bias.

\item The block cipher size divided by $\cdot 5$ must be a perfect square (for
it to be used in a square matrix).

\item The block cipher size must provide the same key and message space
as 256 bits or more.

\end{enumerate}

The solution for the first two conditions are of the form, $3\cdot 5k^2$ for some integer $k$. Here
are the lists for the number of bits we get with $3\cdot 5k^2$ as the quin size.


\begin{enumerate}
\item $k=1, 3\cdot 5(1)^1\log_2(5) \approx 34.82$
\item $k=2, 3\cdot 5(2)^2\log_2(5) \approx 139.31$
\item $k=3, 3\cdot 5(3)^2\log_2(5) \approx 313.46$
\end{enumerate}

So by choosing $3\cdot 5(3)^2 = 135$ as our block size, we get about 313.46
bits of key/message space, which is more than enough.

\subsection{State}

The state is a 3 by 3 matrix with each cells containing 15 quins, achieving
the 135 quins which the cipher operates on:

\[
\textrm{State} := 
\begin{bmatrix}
q_0 & q_1 & q_2 \\
q_3 & q_4 & q_5 \\
q_6 & q_7 & q_8
\end{bmatrix}
\]

where $q_n \in F_5[x]/\langle d \rangle,\mathrm{deg}(d) = 15$ (the field
which contains $5^{15}$ elements with $d$ as the modulus irreducible
polymomial.)


\subsection{Add Round Key}

\textbf{Inputs:} 135 quins state and 135 quins round key.

\noindent \textbf{Output:} 135 quins state.

\medskip

If:

\[
\textrm{State} := 
\begin{bmatrix}
q_0 & q_1 & q_2 \\
q_3 & q_4 & q_5 \\
q_6 & q_7 & q_8
\end{bmatrix}
\]

and

\[
\textrm{RoundKey} := 
\begin{bmatrix}
k_0 & k_1 & k_2 \\
k_3 & k_4 & k_5 \\
k_6 & k_7 & k_8
\end{bmatrix}
\]


then,

\[
\textrm{\textbf{AddRoundKey}(State,RoundKey)} =
\begin{bmatrix}
q_0 + k_0 & q_1 + k_1 & q_2 + k_2 \\
q_3 + k_3 & q_4 + k_4 & q_5 + k_5 \\
q_6 + k_6 & q_7 + k_7 & q_8 + k_8
\end{bmatrix}
\]

where addition of $q_n+k_n$ is performed over $F_{5^{15}}$.

\subsection{Substitute Columns}

\textbf{Inputs:} 135 quins state.

\noindent \textbf{Output:} 135 quins state.

\medskip

To propagate diffusion between rows, each of the 3 columns in the matrix
are treated as element in $F_{5^{45}}$. If $c_n$ is the nth column,
then it's substituted with $S(c_n)$ such that:

\[ S(c) = \begin{cases}
\gamma & c = 0\\
\gamma + 1/c & c\neq 0
\end{cases}
\],

where $\gamma$ is chosen so that $S(c)$ has no fixed point, i.e, $S(c) \neq c$ and no
opposite fixed point, i.e, $S(-c) \neq c$. The Sbox $S(c)$ must also provide
non-linearity, i.e, $S(a + b) \neq S(a) + S(b)$.

If

\[
\textrm{State} = 
\begin{bmatrix}
c_0 & c_1 & c_2
\end{bmatrix},
\]

then

\[
\textrm{\textbf{SubstituteColumns}(State)} =
\begin{bmatrix}
S(c_0) & S(c_1) & S(c_2)
\end{bmatrix},
\]

\subsection{Shift Rows}

\textbf{Inputs:} 135 quin State.

\noindent \textbf{Output:} 135 quins State.

\medskip

To propagate diffusion between columns, each rows are circularly
shifted to the left. The 0th row shifted by 0, the 1st is shifted by 1,
and the 2nd is shifted by 2.

If:

\[
\textrm{State} := 
\begin{bmatrix}
k_0 & k_1 & k_2 \\
k_3 & k_4 & k_5 \\
k_6 & k_7 & k_8
\end{bmatrix},
\]

then

\[
\textrm{\textbf{ShiftRows}(State)} =
\begin{bmatrix}
k_0 & k_1 & k_2 \\
k_4 & k_5 & k_3 \\
k_8 & k_6 & k_7
\end{bmatrix},
\]

\subsection{Round Function}

\textbf{Inputs:} 135 quins State and 135 quins Key

\noindent \textbf{Output:} 135 quins State.

\[
  \textrm{\textbf{RoundFunction}(State, Key)} = \textrm{\textbf{SR}(\textbf{SC}(\textbf{ARK}(State, Key)))}
\]

where \textbf{SR}, \textbf{SC}, \textbf{ARK}, corresponds to \textbf{ShiftRows}, \textbf{SubstituteColumns}, and \textbf{AddRoundKey} respectively.

\subsection{The TQC-135 cipher}

An N-round TQC, consist of N RoundFunctions followed by a \textbf{ARK}, \textbf{SC} and \textbf{ARK},
with RoundKeys generated from a Key Scheduler.

\subsection{Reference implementation in C}

\url{https://github.com/harieamjari/quin/blob/master/quin.tex}

\appendix

\section{GSM 3-Quin default character set}

\label{a1}

This table defines the codepoints in quinary for the 125 characters of this
SMS quinary cipher. This character set is a subset of the GSM 7-bit default
character set defined in 3GPP TS 23.038.

The quinaries are ordered this way in little endian to make it easier to parse the
quinaries left to right.

\begin{longtable}{|l|l|c|c|l|l|c|c|l|l|c|}
  \cline{1-3}
  \cline{5-7}
  \cline{9-11}
  Dec & Quin & Char & & Dec & Quin& Char& & Dec& Quin& Char\\
  \cline{1-3}
  \cline{5-7}
  \cline{9-11}
  0 & 000 & @ & & 42 & 231 & * & & 84 & 413 & T \\
  \cline{1-3}
  \cline{5-7}
  \cline{9-11}
  1 & 100 & £ & & 43 & 331 & + & & 85 & 023 & U \\
  \cline{1-3}
  \cline{5-7}
  \cline{9-11}
  2 & 200 & \$ & & 44 & 431 & , & & 86 & 123 & V \\
  \cline{1-3}
  \cline{5-7}
  \cline{9-11}
  3 & 300 & ¥ & & 45 & 041 & - & & 87 & 223 & W \\
  \cline{1-3}
  \cline{5-7}
  \cline{9-11}
  4 & 400 & è & & 46 & 141 & . & & 88 & 323 & X \\
  \cline{1-3}
  \cline{5-7}
  \cline{9-11}
  5 & 010 & é & & 47 & 241 & / & & 89 & 423 & Y \\
  \cline{1-3}
  \cline{5-7}
  \cline{9-11}
  6 & 110 & ù & & 48 & 341 & 0 & & 90 & 033 & Z \\
  \cline{1-3}
  \cline{5-7}
  \cline{9-11}
  7 & 210 & ì & & 49 & 441 & 1 & & 91 & 133 & Ä \\
  \cline{1-3}
  \cline{5-7}
  \cline{9-11}
  8 & 310 & ò & & 50 & 002 & 2 & & 92 & 233 & Ö \\
  \cline{1-3}
  \cline{5-7}
  \cline{9-11}
  9 & 410 & Ç & & 51 & 102 & 3 & & 93 & 333 & Ñ \\
  \cline{1-3}
  \cline{5-7}
  \cline{9-11}
  10 & 020 & ñ & & 52 & 202 & 4 & & 94 & 433 & Ü \\
  \cline{1-3}
  \cline{5-7}
  \cline{9-11}
  11 & 120 & Ø & & 53 & 302 & 5 & & 95 & 043 & § \\
  \cline{1-3}
  \cline{5-7}
  \cline{9-11}
  12 & 220 & ø & & 54 & 402 & 6 & & 96 & 143 & ¿ \\
  \cline{1-3}
  \cline{5-7}
  \cline{9-11}
  13 & 320 & ü & & 55 & 012 & 7 & & 97 & 243 & a \\
  \cline{1-3}
  \cline{5-7}
  \cline{9-11}
  14 & 420 & Å & & 56 & 112 & 8 & & 98 & 343 & b \\
  \cline{1-3}
  \cline{5-7}
  \cline{9-11}
  15 & 030 & å & & 57 & 212 & 9 & & 99 & 443 & c \\
  \cline{1-3}
  \cline{5-7}
  \cline{9-11}
  16 & 130 & $\Delta$ & & 58 & 312 & : & & 100 &004 & d \\
  \cline{1-3}
  \cline{5-7}
  \cline{9-11}
  17 & 230 & \_ & & 59 & 412 & ; & & 101 & 104 & e \\
  \cline{1-3}
  \cline{5-7}
  \cline{9-11}
  18 & 330 & $\Phi$ & & 60 & 022 & < & & 102 & 204 & f \\
  \cline{1-3}
  \cline{5-7}
  \cline{9-11}
  19 & 430 & $\Gamma$ & & 61 & 122 & =  & & 103 & 304 & g \\
  \cline{1-3}
  \cline{5-7}
  \cline{9-11}
  20 & 040 & $\Lambda$ & & 62 & 222 & > & & 104 & 404 & h \\
  \cline{1-3}
  \cline{5-7}
  \cline{9-11}
  21 & 140 & $\Omega$ & & 63 & 322 & ? & & 105 & 014 & i \\
  \cline{1-3}
  \cline{5-7}
  \cline{9-11}
  22 & 240 & $\Pi$ & & 64 & 422 & ¡ & & 106 & 114 & j \\
  \cline{1-3}
  \cline{5-7}
  \cline{9-11}
  23 & 340 & $\Psi$ & & 65 & 032 & A & & 107 & 214 & k \\
  \cline{1-3}
  \cline{5-7}
  \cline{9-11}
  24 & 440 & $\Sigma$ & & 66 & 132 & B & & 108 & 314 & l \\
  \cline{1-3}
  \cline{5-7}
  \cline{9-11}
  25 & 001 & $\Theta$ & & 67 & 232 & C & & 109 & 414 & m \\
  \cline{1-3}
  \cline{5-7}
  \cline{9-11}
  26 & 101 & $\Xi$ & & 68 & 332 & D & & 110 & 024 & n \\
  \cline{1-3}
  \cline{5-7}
  \cline{9-11}
  27 & 201 & à & & 69 & 432 & E & & 111 & 124 & o \\
  \cline{1-3}
  \cline{5-7}
  \cline{9-11}
  28 & 301 & Æ & & 70 & 042 & F & & 112 & 224  & p \\
  \cline{1-3}
  \cline{5-7}
  \cline{9-11}
  29 & 401 & æ & & 71 & 142 & G & & 113 & 324 & q \\
  \cline{1-3}
  \cline{5-7}
  \cline{9-11}
  30 & 011 & ß & & 72 & 242 & H & & 114 & 424 & r \\
  \cline{1-3}
  \cline{5-7}
  \cline{9-11}
  31 & 111 & É & & 73 & 342 & I & & 115 & 034 & s \\
  \cline{1-3}
  \cline{5-7}
  \cline{9-11}
  32 & 211 & \verb+SP+ & & 74 & 442 & J & & 116 & 134 & t \\
  \cline{1-3}
  \cline{5-7}
  \cline{9-11}
  33 & 311 & ! & & 75 & 003 & K & & 117 & 234 & u \\
  \cline{1-3}
  \cline{5-7}
  \cline{9-11}
  34 & 411 & " & & 76 & 103 & L & & 118 & 334 & v \\
  \cline{1-3}
  \cline{5-7}
  \cline{9-11}
  35 & 021 & \# & & 77 & 203 & M & & 119 & 434 & w \\
  \cline{1-3}
  \cline{5-7}
  \cline{9-11}
  36 & 121 & ¤ & & 78 & 303 & N & & 120 & 044 & x \\
  \cline{1-3}
  \cline{5-7}
  \cline{9-11}
  37 & 221 & \% & & 79 & 403 & O & & 121 & 144 & y \\
  \cline{1-3}
  \cline{5-7}
  \cline{9-11}
  38 & 321 & \& & & 80 & 013 & P & & 122 & 244 & z \\
  \cline{1-3}
  \cline{5-7}
  \cline{9-11}
  39 & 421 & \textquotesingle & & 81 & 113 & Q & & 123 & 344 & ä \\
  \cline{1-3}
  \cline{5-7}
  \cline{9-11}
  40 & 031 & ( & & 82 & 213 & R & & 124 & 444 & ö \\
  \cline{1-3}
  \cline{5-7}
  \cline{9-11}
  41 & 131 & )  & & 83 & 313 & S & & 125 & \cellcolor{black!25}& \cellcolor{black!25}\\
  \cline{1-3}
  \cline{5-7}
  \cline{9-11}
\end{longtable}%
\medskip

\begin{itemize}
\item Dec -- Decimal representation
\item Quin -- Quinary representation
\item Char -- Character representation
\end{itemize}
%
%
\medskip

\medskip \textbf{NOTE} Grey cells are unused.

%\include{arithmetic.tex}
\section*{Postface}

So far, this is all purely theoretical. Yes, I managed to send and
receive messages using these techniques and verified that my phone and
network does not replace sequences of whitespaces to a single one and
they did supported the 125 characters, but I have yet to test these on
other countries with different carriers. If it did work, yey! Let me know!

But would it really be worth to design a whole new secure cryptographic
quinary cipher just for the purpose of SMS encryption only? Probably,
no, but atleast it has satisfied me.

Directly from the tropics, over $30^\circ$C.

\section*{Acknowledgement}


I thank whole-heartedly Sean Erik O'Connor and the contributors for their work in the Primpoly software which helped accelerate the finding
of primitive polynomials for use in this article.

\end{document}

